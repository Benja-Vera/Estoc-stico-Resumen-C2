\documentclass{article}
\usepackage[letterpaper, margin = 1cm]{geometry}
\usepackage[utf8]{inputenc}
\usepackage[english,spanish]{babel}
\usepackage{csquotes}
\usepackage{fancyhdr}
\usepackage[dvipsnames]{xcolor}
\usepackage{amssymb}
\usepackage{amsmath}
\usepackage{fontawesome}
\usepackage[unicode]{hyperref}
\usepackage{blindtext}
\usepackage{multicol}
\usepackage{caption}
\usepackage{physics}
\usepackage{graphicx}
\usepackage[breakable]{tcolorbox}
\usepackage{fourier}
\usepackage{enumitem}
% Alfabeto
\newcommand{\A}{\mathcal{A}}
\newcommand{\B}{\mathcal{B}}
\newcommand{\C}{\mathbb{C}}
\newcommand{\D}{\mathcal{D}}
\newcommand{\E}{\mathbb{E}}
\newcommand{\F}{\mathcal{F}}
\newcommand{\I}{\mathcal{I}}
\newcommand{\K}{\mathcal{K}}
\renewcommand{\L}{\mathcal{L}}
\newcommand{\M}{\mathcal{M}}
\newcommand{\N}{\mathbb{N}}
\renewcommand{\P}{\mathbb{P}}
\newcommand{\Q}{\mathbb{Q}}
\newcommand{\R}{\mathbb{R}}
\renewcommand{\S}{\mathcal{S}}
\newcommand{\T}{\mathcal{T}}
\newcommand{\Z}{\mathbb{Z}}

% Griego
\renewcommand{\epsilon}{\varepsilon}

% GEOMETRY
\setlength{\headheight}{14pt}
\setlength{\parskip}{0em}
\pagestyle{empty}
% Columnas
\setlength{\columnsep}{2.5em}
% Enumitem
\setlist{nosep}

% TIKZ STUFF
\definecolor{dBlue}{RGB}{62, 71, 151}
\definecolor{lGreen}{RGB}{137, 190, 111}
\definecolor{lBlue}{RGB}{140, 208, 242}
\definecolor{orange}{RGB}{239, 147, 85}
\definecolor{black}{RGB}{0, 0, 0}
\definecolor{red}{RGB}{206, 52, 52}
\definecolor{grey}{RGB}{128, 128, 128}
\definecolor{dGreen}{RGB}{0, 140, 0}
\definecolor{purple}{RGB}{125, 0, 170}
\definecolor{pink}{RGB}{173, 109, 168}
\definecolor{lavander}{RGB}{101, 94, 163}
\definecolor{cyan}{RGB}{0, 157, 169}


% hyperref
\hypersetup{
    colorlinks=true,
    linkcolor=blue,
    filecolor=magenta,  
    urlcolor=black,
    citecolor=dGreen,
}

% Producto interno
\renewcommand{\innerproduct}[2]{\langle #1, #2 \rangle}
% Independencia de VAs
\newcommand{\ind}{\perp\!\!\!\!\perp} 
% Operadores
\DeclareMathOperator{\Var}{Var}
\DeclareMathOperator{\Cov}{Cov}
\newcommand{\floor}[1]{\lfloor #1 \rfloor}

% Título
\newcommand{\titulo}[1]
{\thispagestyle{plain}
\begin{center}
    \huge
    \textbf{#1}\\
    \normalsize
    \textit{MA5402 Cálculo Estocástico - Primavera 2023}
\end{center}}


% Definición
\newcommand{\definicion}[2]
{\noindent\makebox[0pt][r]{\faBookmark\quad}\textbf{Definición: [#1]} #2}


% Proposición
\newcommand{\proposicion}[2]
{\noindent\makebox[0pt][r]{\faLightbulbO\quad}\textbf{Proposición: [#1]} #2}

% Lema
\newcommand{\lema}[2]
{\noindent\makebox[0pt][r]{\faGear\quad}\textbf{Lema: [#1]} #2}

% Teorema
\newcommand{\teorema}[2]
{\noindent\makebox[0pt][r]{\faBook\quad}\textbf{Teorema: [#1]} #2}

\begin{document}
\titulo{Resumen Control 2}

\begin{multicols}{2}
\setcounter{section}{-1}
\section{Preliminares}
\definicion{Nociones de convergencia}{
    LAL
}

\definicion{Funciones de variación acotada}{
    LEL
}

\definicion{Esperanza Condicional}{
    Para $(\Omega, \F, \P)$ espacio de probabilidad, $X$ variable aleatoria en $L^1$ y $\G$ sub $\sigma$-álgebra de $\F$, la esperanza condicional $\E[X | \G]$ se define como aquella variable integrable $\G$-medible tal que
    \[\forall G \in \G : \E[Y \mathbf{1}_{G}] = \E[X \mathbf{1}_{G}]\]
    La esperanza condicional posee, entre otras, las siguientes propiedades:
    \begin{enumerate}
        \item \textbf{(Esperanza anidada)} $\E[\E[X | \G]] = \E[X]$
        \item \textbf{(Jensen)} Si $c : \R \to \R$ convexa con $c(X)$ integrable, entonces
        \[\E[c(X) | \G] \geq c(E[X | \G])\]
        \item \textbf{(medibilidad)} Si $Z$ es $\G$-medible, entonces
        \[\E[ZX | \G] = Z \E[X | G]\]
        \item \textbf{(independencia)}
    \end{enumerate}
}

\section{Martingalas continuas}

Sea $(\Omega, \F, (\F_t)_{t}, \P)$ un e.d.p filtrado.

\definicion{(sub/super) Martingala}{
Un proceso $(X_t)_{t \geq 0}$ adaptado tal que $X_t \in L^1$ para todo $t \geq 0$ se dice
\begin{enumerate}
    \item \textbf{Martingala} si $\forall 0 \leq s \leq t : \E[X_t | \F_s] = X_s$
    \item \textbf{Supermartingala} si $\forall 0 \leq s \leq t : \E[X_t | \F_s] \leq X_s$
    \item \textbf{Submartingala} si $\forall s \leq s \leq t : \E[X_t | \F_s] \geq X_s$
\end{enumerate}
}

\proposicion{Ejemplos de martingalas}{
Sea $(B_t)_t$ un movimiento browniano, son ejemplos de martingalas los siguientes procesos:
\begin{itemize}
    \item $B_t$
    \item $B_t^2 - t$
    \item $\exp(\theta B_t - \frac{\theta^2}{2}t), \qquad \theta > 0$
\end{itemize}
}

\proposicion{Funciones convexas}{
Sea $(X_t)_t$ adaptado y $f: \R \to \R$ convexa tal que $\E[|f(X_t)|] < \infty$ para todo $t$. Entonces
\begin{itemize}
    \item Si $(X_t)_t$ es martingala, entonces $(f(X_t))_t$ es submartingala.
    \item Si $(X_t)_t$ es submartingala y $f$ es no decreciente, entonces $(f(X_t))_t$ es submartingala.
\end{itemize}
}

\proposicion{supremo}{
Si $(X_t)_t$ es (sub/super) martingala, entonces para todo $t \geq 0$
\[\sup_{s \in [0, t]} \E[|X_s|] < \infty\]}

\teorema{Desigualdades clásicas}{
\begin{enumerate}
    \item \textbf{(Desigualdad maximal)} Sea $(X_t)_t$ supermartingala continua por la derecha. Entonces para $t, \lambda > 0$
    \[\lambda \P\qty[\sup_{s \in [0, t]} |X_s| > \lambda] \leq \E\qty[|X_0|] + 2 \E\qty[|X_t|]\]
    \item \textbf{(Desigualdad de Doob en $L^p$)} Sea $(X_t)_t$ martingala continua por la derecha. Entonces, para $t > 0$, $p > 1$
    \[\E\qty[\sup_{s \in [0, t]} |X_s|^p] \leq \qty(\frac{p}{p-1})^p \E\qty[|X_t|^p]\]
\end{enumerate}
}

\proposicion{Subidas y bajadas}{LIL}

\teorema{De convergencia no UI}{
Sea $X$ sobremartingala continua por la derecha tal que $\sup_{t \geq 0} \E[|x_t|] < \infty$. Entonces, existe $X_\infty \in L^1$ tal que
\[X_t \to X_\infty\]
Cuando $t \to \infty$ en el sentido casi seguro.}

\definicion{Cerradura}{Decimos que $(X_t)_t$ es martingala cerrada si existe $Z \in L^1$ tal que para todo $t \geq 0$
\[X_t = \E[Z | \F_t]\]}
\teorema{De convergencia UI}{Sea $(X_t)_t$ martingala continua por la derecha. LSSE:
\begin{enumerate}
    \item $X$ cerrada.
    \item $X$ UI.
    \item $X$ converge casi seguramente y en $L^1$ a una $X_\infty$ cuando $t \to \infty$.
    En cualquiera de estos casos tenemos que para todo $t \geq 0$
    \[X_t = \E[X_\infty | \F_t]\]
\end{enumerate}}

\teorema{De parada opcional de Doob no acotado}{Sea $(X_t)_t$ martingala UI continua a la derecha. Si $S, T$ son dos t.d.p con $S \leq T$, entonces $X_S, X_T \in L^1$ y
\[X_S = \E[X_T | \F_S]\]
En particular, para $S$ t.d.p, tenemos $X_S = \E[X_\infty | \F_S]$ y tomando esperanza se concluye $\E[X_S] = \E[X_\infty] = \E[X_0]$
}

\teorema{De parada opcional de Doob acotado}{Sea $(X_t)_t$ martingala continua por la derecha y $S \leq T$ t.d.p acotados. Entonces $X_S, X_T \in L^1$ y
\[X_S = \E[X_T | \F_S]\]}

\proposicion{Martingala detenida}{
Sea $(X_t)_t$ martingala continua a la derecha y $T$ t.d.p. Definimos $(X^T_t)_t$ como el proceso detenido $X^T_t = X_{t \wedge t}$. Entonces:
\begin{itemize}
    \item $X^T$ es martingala.
    \item Si $X$ es martingala UI, entonces $X^T$ también lo es y tenemos
    \[X^T_t = \E[X_T | \F_t]\]
\end{itemize}
}

\definicion{Martingala reversa}{LAL}

\teorema{Convergencia de martingalas reversas}{LEL}

\section{Variaciones y Martingalas Locales}


\end{multicols}

\end{document}