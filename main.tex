\documentclass{article}
\usepackage[letterpaper, margin = 1cm]{geometry}
\usepackage[utf8]{inputenc}
\usepackage[english,spanish]{babel}
\usepackage{csquotes}
\usepackage{fancyhdr}
\usepackage[dvipsnames]{xcolor}
\usepackage{amssymb}
\usepackage{amsmath}
\usepackage{fontawesome}
\usepackage[unicode]{hyperref}
\usepackage{blindtext}
\usepackage{multicol}
\usepackage{caption}
\usepackage{physics}
\usepackage{graphicx}
\usepackage[breakable]{tcolorbox}
\usepackage{fourier}
\usepackage{enumitem}
% Alfabeto
\newcommand{\A}{\mathcal{A}}
\newcommand{\B}{\mathcal{B}}
\newcommand{\C}{\mathbb{C}}
\newcommand{\D}{\mathcal{D}}
\newcommand{\E}{\mathbb{E}}
\newcommand{\F}{\mathcal{F}}
\newcommand{\I}{\mathcal{I}}
\newcommand{\K}{\mathcal{K}}
\renewcommand{\L}{\mathcal{L}}
\newcommand{\M}{\mathcal{M}}
\newcommand{\N}{\mathbb{N}}
\renewcommand{\P}{\mathbb{P}}
\newcommand{\Q}{\mathbb{Q}}
\newcommand{\R}{\mathbb{R}}
\renewcommand{\S}{\mathcal{S}}
\newcommand{\T}{\mathcal{T}}
\newcommand{\Z}{\mathbb{Z}}

% Griego
\renewcommand{\epsilon}{\varepsilon}

% GEOMETRY
\setlength{\headheight}{14pt}
\setlength{\parskip}{0em}
\pagestyle{empty}
% Columnas
\setlength{\columnsep}{2.5em}
% Enumitem
\setlist{nosep}

% TIKZ STUFF
\definecolor{dBlue}{RGB}{62, 71, 151}
\definecolor{lGreen}{RGB}{137, 190, 111}
\definecolor{lBlue}{RGB}{140, 208, 242}
\definecolor{orange}{RGB}{239, 147, 85}
\definecolor{black}{RGB}{0, 0, 0}
\definecolor{red}{RGB}{206, 52, 52}
\definecolor{grey}{RGB}{128, 128, 128}
\definecolor{dGreen}{RGB}{0, 140, 0}
\definecolor{purple}{RGB}{125, 0, 170}
\definecolor{pink}{RGB}{173, 109, 168}
\definecolor{lavander}{RGB}{101, 94, 163}
\definecolor{cyan}{RGB}{0, 157, 169}


% hyperref
\hypersetup{
    colorlinks=true,
    linkcolor=blue,
    filecolor=magenta,  
    urlcolor=black,
    citecolor=dGreen,
}

% Producto interno
\renewcommand{\innerproduct}[2]{\langle #1, #2 \rangle}
% Independencia de VAs
\newcommand{\ind}{\perp\!\!\!\!\perp} 
% Operadores
\DeclareMathOperator{\Var}{Var}
\DeclareMathOperator{\Cov}{Cov}
\newcommand{\floor}[1]{\lfloor #1 \rfloor}

% Título
\newcommand{\titulo}[1]
{\thispagestyle{plain}
\begin{center}
    \huge
    \textbf{#1}\\
    \normalsize
    \textit{MA5402 Cálculo Estocástico - Primavera 2023}
\end{center}}


% Definición
\newcommand{\definicion}[2]
{\noindent\makebox[0pt][r]{\faBookmark\quad}\textbf{Definición: [#1]} #2}


% Proposición
\newcommand{\proposicion}[2]
{\noindent\makebox[0pt][r]{\faLightbulbO\quad}\textbf{Proposición: [#1]} #2}

% Lema
\newcommand{\lema}[2]
{\noindent\makebox[0pt][r]{\faGear\quad}\textbf{Lema: [#1]} #2}

% Teorema
\newcommand{\teorema}[2]
{\noindent\makebox[0pt][r]{\faBook\quad}\textbf{Teorema: [#1]} #2}

\begin{document}
\titulo{Resumen Control 2}

\begin{multicols}{2}
\setcounter{section}{-1}
\section{Preliminares}
\definicion{Nociones de convergencia}{Sea $(X_t)_t \in [0, \infty]$ una secuencia de variables aleatorias con dominio en el mismo espacio de probabilidad. Decimos que
\begin{enumerate}
    \item $X_t \to X_\infty$ c.s cuando $\P[\omega : X_n(\omega) \to X_\infty(\omega)] = 1$
    \item $X_t \to X_\infty$ en $L^p$ cuando $\E[|X_t - X_\infty|^p] \to 0$ si $t \to \infty$
    \item $X_t \to X_\infty$ en $\P$ cuando $\forall \epsilon > 0 : \P[|X_t - X_\infty| \geq \epsilon] \to 0$ si $t \to \infty$
\end{enumerate}

Sabemos que 1. y 2. implican 3. También, convergencia en $L^p$ implica convergencia en $L^s$ si $s \in [1, p]$. Además, 3. implica que existe subsucesión que converge c.s. Además, tanto 1. como 3. en conjunto con la existencia de $Y \in L^p$ con $|X_t| \leq Y$ implican 2. Finalmente, tanto 2. como 3. son convergencias metrizables.
}

\definicion{Esperanza Condicional}{
    Para $(\Omega, \F, \P)$ espacio de probabilidad, $X$ variable aleatoria en $L^1$ y $\G$ sub $\sigma$-álgebra de $\F$, la esperanza condicional $\E[X | \G]$ se define como aquella variable integrable $\G$-medible tal que
    \[\forall G \in \G : \E[Y \mathbf{1}_{G}] = \E[X \mathbf{1}_{G}]\]
    La esperanza condicional posee, entre otras, las siguientes propiedades:
    \begin{enumerate}
        \item \textbf{(Esperanza anidada)} $\E[\E[X | \G]] = \E[X]$
        \item \textbf{(Jensen)} Si $c : \R \to \R$ convexa con $c(X)$ integrable, entonces
        \[\E[c(X) | \G] \geq c(E[X | \G])\]
        \item \textbf{(medibilidad)} Si $Z$ es $\G$-medible, entonces
        \[\E[ZX | \G] = Z \E[X | G]\]
        \item \textbf{(independencia)} Si $X$ es independiente de $\G$, entonces
        \[\E[X | \G] = \E[X]\]
    \end{enumerate}}
\section{Martingalas continuas}

Sea $(\Omega, \F, (\F_t)_{t}, \P)$ un e.d.p filtrado.

\definicion{(sub/super) Martingala}{
Un proceso $(X_t)_{t \geq 0}$ adaptado tal que $X_t \in L^1$ para todo $t \geq 0$ se dice
\begin{enumerate}
    \item \textbf{Martingala} si $\forall 0 \leq s \leq t : \E[X_t | \F_s] = X_s$
    \item \textbf{Supermartingala} si $\forall 0 \leq s \leq t : \E[X_t | \F_s] \leq X_s$
    \item \textbf{Submartingala} si $\forall s \leq s \leq t : \E[X_t | \F_s] \geq X_s$
\end{enumerate}
}

\proposicion{Ejemplos de martingalas}{
Sea $(B_t)_t$ un movimiento browniano, son ejemplos de martingalas los siguientes procesos:
\begin{itemize}
    \item $B_t$
    \item $B_t^2 - t$
    \item $\exp(\theta B_t - \frac{\theta^2}{2}t), \qquad \theta > 0$
\end{itemize}
}

\proposicion{Funciones convexas}{
Sea $(X_t)_t$ adaptado y $f: \R \to \R$ convexa tal que $\E[|f(X_t)|] < \infty$ para todo $t$. Entonces
\begin{itemize}
    \item Si $(X_t)_t$ es martingala, entonces $(f(X_t))_t$ es submartingala.
    \item Si $(X_t)_t$ es submartingala y $f$ es no decreciente, entonces $(f(X_t))_t$ es submartingala.
\end{itemize}
}

\proposicion{supremo}{
Si $(X_t)_t$ es (sub/super) martingala, entonces para todo $t \geq 0$
\[\sup_{s \in [0, t]} \E[|X_s|] < \infty\]}

\teorema{Desigualdades clásicas}{
\begin{enumerate}
    \item \textbf{(Desigualdad maximal)} Sea $(X_t)_t$ supermartingala continua por la derecha. Entonces para $t, \lambda > 0$
    \[\lambda \P\qty[\sup_{s \in [0, t]} |X_s| > \lambda] \leq \E\qty[|X_0|] + 2 \E\qty[|X_t|]\]
    \item \textbf{(Desigualdad de Doob en $L^p$)} Sea $(X_t)_t$ martingala continua por la derecha. Entonces, para $t > 0$, $p > 1$
    \[\E\qty[\sup_{s \in [0, t]} |X_s|^p] \leq \qty(\frac{p}{p-1})^p \E\qty[|X_t|^p]\]
\end{enumerate}
}

\lema{Subidas y bajadas}{El número $\gamma_{a, b}(t)$ de subidas y bajadas de $[a, b]$ hecho por $t \mapsto X_t(\omega)$ hasta el tiempo $t$ se define como el mayor $k \in \N$ tal que existen
\[0 \leq s_1 < t_1 < s_2 < t_2 < \dots < s_k < t_k \leq t\]
con $X_{s_i}(\omega) < a$ y $X_{t_i}(\omega) > b$. Con esto, tenemos que
\[\E[\gamma_{a, b}(t)] \leq \frac{1}{b-a} \E[(X_t - a)^-]\]}

\definicion{Uniforme integrabilidad}{Decimos que $(X_t)_t$ es uniformemente integrable (UI) si cuando $a \to \infty$,
\[\sup_{t \geq 0} \E[|X_t| \mathbf{1}_{|x_t| \geq a}] \to 0\]}

\proposicion{Condiciones para ser UI}{$(X_t)_t$ es UI si se tiene alguna de estas condiciones
\begin{itemize}
    \item $(|X_t|)_t$ está acotado por una v.a en $L^1$ o por un proceso UI.
    \item Existe $\phi: \R^+ \to \R$ convexa tal que
    \[\frac{\phi(x)}{x} \to \infty,\qquad \sup_{t \geq 0} \E[\phi(|x_t|)] < \infty\]
\end{itemize}}

\teorema{De convergencia no UI}{
Sea $X$ supermartingala continua por la derecha tal que $\sup_{t \geq 0} \E[|x_t|] < \infty$. Entonces, existe $X_\infty \in L^1$ tal que
\[X_t \to X_\infty\]
Cuando $t \to \infty$ en el sentido casi seguro.}

\definicion{Cerradura}{Decimos que $(X_t)_t$ es martingala cerrada si existe $Z \in L^1$ tal que para todo $t \geq 0$
\[X_t = \E[Z | \F_t]\]}
\teorema{De convergencia UI}{Sea $(X_t)_t$ martingala continua por la derecha. LSSE:
\begin{enumerate}
    \item $X$ cerrada.
    \item $X$ UI.
    \item $X$ converge casi seguramente y en $L^1$ a una $X_\infty$ cuando $t \to \infty$.
    En cualquiera de estos casos tenemos que para todo $t \geq 0$
    \[X_t = \E[X_\infty | \F_t]\]
\end{enumerate}}

\teorema{De parada opcional de Doob no acotado}{Sea $(X_t)_t$ martingala UI continua a la derecha. Si $S, T$ son dos t.d.p con $S \leq T$, entonces $X_S, X_T \in L^1$ y
\[X_S = \E[X_T | \F_S]\]
En particular, para $S$ t.d.p, tenemos $X_S = \E[X_\infty | \F_S]$ y tomando esperanza se concluye $\E[X_S] = \E[X_\infty] = \E[X_0]$
}

\teorema{De parada opcional de Doob acotado}{Sea $(X_t)_t$ martingala continua por la derecha y $S \leq T$ t.d.p acotados. Entonces $X_S, X_T \in L^1$ y
\[X_S = \E[X_T | \F_S]\]
Tomando esperanza y $S = 0$, se obtiene $\E[X_T] = \E[X_0]$}

\proposicion{Martingala detenida}{
Sea $(X_t)_t$ (super)martingala continua a la derecha y $T$ t.d.p. Definimos $(X^T_t)_t$ como el proceso detenido $X^T_t = X_{t \wedge t}$. Entonces:
\begin{itemize}
    \item $X^T$ es (super)martingala.
    \item Si $X$ es martingala UI, entonces $X^T$ también lo es y tenemos
    \[X^T_t = \E[X_T | \F_t]\]
\end{itemize}
}

Lo siguiente solo aplica para el caso discreto

\definicion{Martingala reversa}{
Un proceso indexado por los enteros negativos $(X_n)_{n \in -\N}$ se dice martingala reversa si $X_0 \in L^1$ y para todo $m < n$ se cumple
\[\E[X_n | \F_m] = X_m\]
Aquí $(\F_n)_n$ es un filtración reversa, en el sentido de que es decreciente hacia los negativos.
}

\teorema{Convergencia de martingalas reversas}{Sea $X$ una martingala reversa. Entonces existe $X_{-\infty}$ tal que $X_n \to X_{-\infty}$ c.s y en $L^1$. Si definimos $\F_{-\infty} = \bigcap_{n \in -\N} \F_n$, entonces
\[X_{-\infty} = \E[X_0 | \F_{-\infty}]\]}

\section{Variaciones y Martingalas Locales}

\definicion{Variación}{
Para $f: [0, T] \to \R$, definimos
\[V_T(f) = \sup\qty{\sum_{i = 0}^{N-1} |f(t_{i+1}) - f(t_i)|}\]
En que el supremo se toma sobre todas las particiones finitas del intervalo $[0, T]$. Si $V_T(f) < \infty$ para todo $T \geq 0$, decimos que $f$ es de variación acotada.
}

\proposicion{Descomposición}{$A$ es función de variación acotada ssi se puede descomponer como suma de una función creciente y otra decreciente.

A partir de una función de variación acotada, se puede definir una medida con signo finita $\mu$ sobre intervalos acotados. Con lo cual se define
\[\int_0^T f(s) \dd{A}(s) = \int_{[0, T]} f(s) \dd{\mu}(s)\]}

\proposicion{Continuidad de la variación}{Si $f$ es una función continua, $t \mapsto V_t(f)$ es una función continua.}

\lema{De aproximación}{
Sea $f: [0, T] \to \R$ continua y $A: [0, T] \to \R$ continua de variación acotada. Si $(t_i^n)_{i = 0}^n$ es una secuencia de particiones de $[0, T]$ con $t_0^n = 0$ y $t_n^n = T$ y tales que $\Delta t_i^n \to 0$ cuando $n \to \infty$, entonces
\[\int_0^T f(s) \dd{A}(s) = \lim_{n \to \infty} \sum_{i = 0}^{n - 1} f(t_i)(A(t_{i + 1}^n) - A(t_i^n))\]
}

\teorema{Martingalas de variación acotada}{
Si $M$ es martingala de variación acotada con $M_0 = 0$ c.s, entonces $M \equiv 0$.
}

\definicion{Martingala local}{Decimos que $(M_t)_t$ es martingala local continua que parte de $0$ si
\begin{enumerate}
    \item $M_0 = 0$
    \item Existe $(\tau_n)_n \nearrow \infty$ con $M^{\tau_n}$ martingala continua UI (decimos en este caso que la sucesión $(\tau_n)_n$ \textit{localiza} a $M$)
\end{enumerate}
Más generalmente, decimos que $(M_t)_t$ es martingala local continua si $(M_t - M_0)_t$ es una martingala local que parte de $0$ con $M_0$ $\F_0$-medible.}

\proposicion{Sobre tiempos de localización}{
Si $(M_t)_t$ es martingala local continua y
\[\tau_n = \inf\qty{ t \geq 0 : M_t \notin (-n, n)}\]
Entonces $(\tau_n)_n$ localiza a $M$.
}

\proposicion{Condiciones para que una martingala local sea martingala}{Sea $(M_t)_t$ martingala local continua con $M_0$ integrable.
\begin{itemize}
    \item Si $M$ es no-negativa, entonces es supermartingala.
    \item Si existe $Y \in L^1$ tal que $\sup_{t \geq 0} |M_t| \leq Y$, entonces $M$ es martingala UI.
    \item Si $\tau$ es t.d.p, entonces $M^\tau$ es una martingala local.
\end{itemize}
}

\teorema{Martingalas locales de variación acotada}{Si $M$ es martingala local que parte en $0$ de variación acotada, entonces $M \equiv 0$.}

\definicion{Variación cuadrática para martingalas locales}{
    Sea $(M_t)_{t\geq 0}$ una martingala local continua, entonces existe un único proceso $\langle M\rangle_t$ tal que 
    \begin{enumerate}
        \item $\langle M\rangle_0 = 0$
        \item $\langle M\rangle_t$ es creciente
        \item $M^2_t - \langle M\rangle_t $ es una martingala local continua
    \end{enumerate}
    Más aún, si fijamos $T>0$, $((t_i^n)_{i=1}^n)_{n\in \N}$ una secuencia (determinista) de particiones con $t_0^n = 0$, $t_n^n = T$ y $\Delta t_i^n \rightarrow 0$.
    \[\langle M\rangle_T = \lim_{n \to \infty} \sum_{i=0}^{n-1} (M(t_{i+1}^n) -M(t_i))^2\]
    en probabilidades. Al proceso $(\langle M\rangle_t)_{t\geq 0}$ lo llamamos variación cuadrática de $M$.
    
    Si $M$ es además una martingala acotada (c.s. por una constante determinista), entonces $M^2_t - \langle M\rangle_t$ es una martingala.
}


\end{multicols}

\end{document}