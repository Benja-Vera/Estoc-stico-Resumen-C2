% Alfabeto
\newcommand{\A}{\mathcal{A}}
\newcommand{\B}{\mathcal{B}}
\newcommand{\C}{\mathbb{C}}
\newcommand{\D}{\mathcal{D}}
\newcommand{\E}{\mathbb{E}}
\newcommand{\F}{\mathcal{F}}
\newcommand{\G}{\mathcal{G}}
\newcommand{\I}{\mathcal{I}}
\newcommand{\K}{\mathcal{K}}
\renewcommand{\L}{\mathcal{L}}
\newcommand{\M}{\mathcal{M}}
\newcommand{\N}{\mathbb{N}}
\renewcommand{\P}{\mathbb{P}}
\newcommand{\Q}{\mathbb{Q}}
\newcommand{\R}{\mathbb{R}}
\renewcommand{\S}{\mathcal{S}}
\newcommand{\T}{\mathcal{T}}
\newcommand{\Z}{\mathbb{Z}}

% Griego
\renewcommand{\epsilon}{\varepsilon}

% GEOMETRY
\setlength{\headheight}{14pt}
\setlength{\parskip}{0em}
\pagestyle{empty}
% Columnas
\setlength{\columnsep}{2.5em}
% Enumitem
\setlist{nosep}

% TIKZ STUFF
\definecolor{dBlue}{RGB}{62, 71, 151}
\definecolor{lGreen}{RGB}{137, 190, 111}
\definecolor{lBlue}{RGB}{140, 208, 242}
\definecolor{orange}{RGB}{239, 147, 85}
\definecolor{black}{RGB}{0, 0, 0}
\definecolor{red}{RGB}{206, 52, 52}
\definecolor{grey}{RGB}{128, 128, 128}
\definecolor{dGreen}{RGB}{0, 140, 0}
\definecolor{purple}{RGB}{125, 0, 170}
\definecolor{pink}{RGB}{173, 109, 168}
\definecolor{lavander}{RGB}{101, 94, 163}
\definecolor{cyan}{RGB}{0, 157, 169}


% hyperref
\hypersetup{
    colorlinks=true,
    linkcolor=blue,
    filecolor=magenta,  
    urlcolor=black,
    citecolor=dGreen,
}

% Producto interno
\newcommand{\qcovariation}[2]{\langle #1, #2 \rangle}
% Variación cuadrática
\newcommand{\qvariation}[1]{\langle #1 \rangle}
% Independencia de VAs
\newcommand{\ind}{\perp\!\!\!\!\perp} 
% Operadores
\DeclareMathOperator{\Var}{Var}
\DeclareMathOperator{\Cov}{Cov}
\newcommand{\floor}[1]{\lfloor #1 \rfloor}

% Título
\newcommand{\titulo}[1]
{\thispagestyle{plain}
\begin{center}
    \huge
    \textbf{#1}\\
    \normalsize
    \textit{MA5402 Cálculo Estocástico - Primavera 2023}
\end{center}}


% Definición
\newcommand{\definicion}[2]
{\noindent\makebox[0pt][r]{\faBookmark\quad}\textbf{Definición: [#1]} #2}


% Proposición
\newcommand{\proposicion}[2]
{\noindent\makebox[0pt][r]{\faLightbulbO\quad}\textbf{Proposición: [#1]} #2}

% Lema
\newcommand{\lema}[2]
{\noindent\makebox[0pt][r]{\faGear\quad}\textbf{Lema: [#1]} #2}

% Teorema
\newcommand{\teorema}[2]
{\noindent\makebox[0pt][r]{\faBook\quad}\textbf{Teorema: [#1]} #2}